\documentclass[a4paper, 11pt, oneside]{article}

\usepackage[utf8]{inputenc}
\usepackage[T1]{fontenc}
\usepackage[french]{babel}
\usepackage{array}
\usepackage{shortvrb}
\usepackage{listings}
\usepackage[fleqn]{amsmath}
\usepackage{amsfonts}
\usepackage{fullpage}
\usepackage{enumerate}
\usepackage{graphicx}             % import, scale, and rotate graphics
\usepackage{subfigure}            % group figures
\usepackage{alltt}
\usepackage{url}
\usepackage{indentfirst}
\usepackage{eurosym}
\usepackage{listings}
\usepackage{color}
\usepackage[table,xcdraw,dvipsnames]{xcolor}

% Change le nom par défaut des listing
\renewcommand{\lstlistingname}{Extrait de Code}

% Change la police des titres pour convenir à votre seul lecteur
\usepackage{sectsty}
\allsectionsfont{\sffamily\mdseries\upshape} 
% Idem pour la table des matière.
\usepackage[nottoc,notlof,notlot]{tocbibind} 
\usepackage[titles,subfigure]{tocloft} 
\renewcommand{\cftsecfont}{\rmfamily\mdseries\upshape}
\renewcommand{\cftsecpagefont}{\rmfamily\mdseries\upshape} 

\definecolor{mygray}{rgb}{0.5,0.5,0.5}
\newcommand{\coms}[1]{\textcolor{MidnightBlue}{#1}}

\lstset{
    language=C, % Utilisation du langage C
    commentstyle={\color{MidnightBlue}}, % Couleur des commentaires
    frame=single, % Entoure le code d'un joli cadre
    rulecolor=\color{black}, % Couleur de la ligne qui forme le cadre
    stringstyle=\color{RawSienna}, % Couleur des chaines de caractères
    numbers=left, % Ajoute une numérotation des lignes à gauche
    numbersep=5pt, % Distance entre les numérots de lignes et le code
    numberstyle=\tiny\color{mygray}, % Couleur des numéros de lignes
    basicstyle=\tt\footnotesize, 
    tabsize=3, % Largeur des tabulations par défaut
    keywordstyle=\tt\bf\footnotesize\color{Sepia}, % Style des mots-clés
    extendedchars=true, 
    captionpos=b, % sets the caption-position to bottom
    texcl=true, % Commentaires sur une ligne interprétés en Latex
    showstringspaces=false, % Ne montre pas les espace dans les chaines de caractères
    escapeinside={(>}{<)}, % Permet de mettre du latex entre des <( et )>.
    inputencoding=utf8,
    literate=
  {á}{{\'a}}1 {é}{{\'e}}1 {í}{{\'i}}1 {ó}{{\'o}}1 {ú}{{\'u}}1
  {Á}{{\'A}}1 {É}{{\'E}}1 {Í}{{\'I}}1 {Ó}{{\'O}}1 {Ú}{{\'U}}1
  {à}{{\`a}}1 {è}{{\`e}}1 {ì}{{\`i}}1 {ò}{{\`o}}1 {ù}{{\`u}}1
  {À}{{\`A}}1 {È}{{\`E}}1 {Ì}{{\`I}}1 {Ò}{{\`O}}1 {Ù}{{\`U}}1
  {ä}{{\"a}}1 {ë}{{\"e}}1 {ï}{{\"i}}1 {ö}{{\"o}}1 {ü}{{\"u}}1
  {Ä}{{\"A}}1 {Ë}{{\"E}}1 {Ï}{{\"I}}1 {Ö}{{\"O}}1 {Ü}{{\"U}}1
  {â}{{\^a}}1 {ê}{{\^e}}1 {î}{{\^i}}1 {ô}{{\^o}}1 {û}{{\^u}}1
  {Â}{{\^A}}1 {Ê}{{\^E}}1 {Î}{{\^I}}1 {Ô}{{\^O}}1 {Û}{{\^U}}1
  {œ}{{\oe}}1 {Œ}{{\OE}}1 {æ}{{\ae}}1 {Æ}{{\AE}}1 {ß}{{\ss}}1
  {ű}{{\H{u}}}1 {Ű}{{\H{U}}}1 {ő}{{\H{o}}}1 {Ő}{{\H{O}}}1
  {ç}{{\c c}}1 {Ç}{{\c C}}1 {ø}{{\o}}1 {å}{{\r a}}1 {Å}{{\r A}}1
  {€}{{\euro}}1 {£}{{\pounds}}1 {«}{{\guillemotleft}}1
  {»}{{\guillemotright}}1 {ñ}{{\~n}}1 {Ñ}{{\~N}}1 {¿}{{?`}}1
}
\newcommand{\tablemat}{~}

%%%%%%%%%%%%%%%%% TITRE %%%%%%%%%%%%%%%%
\newcommand{\intitule}{Projet 2}
\newcommand{\GrNbr}{10}
\newcommand{\PrenomUN}{Cyril}
\newcommand{\NomUN}{RUSSE}
\renewcommand{\tablemat}{\tableofcontents}

%%%%%%%% ZONE PROTÉGÉE : MODIFIEZ UNE DES DIX PROCHAINES %%%%%%%%
%%%%%%%%            LIGNES POUR PERDRE 2 PTS.            %%%%%%%%
\title{INFO0947: \intitule\\ Types Abstraits de Données}
\author{Groupe \GrNbr : \PrenomUN~\textsc{\NomUN}}
\date{}
\begin{document}
\maketitle
\newpage
\tablemat
\newpage
%%%%%%%%%%%%%%%%%%%% FIN DE LA ZONE PROTÉGÉE %%%%%%%%%%%%%%%%%%%%

%%%%%%%%%%%%%%%% RAPPORT %%%%%%%%%%%%%%%

\section{\textbf{Introduction}}

\subsection{\textbf{Contexte}}

En référence à la BD Astérix et Obelix, "Le tour de Gaule d'Astérix", ce travail a pour 
but d'implémenter ce fameux tour sous la forme d'un type abstrait de données.

\subsection{\textbf{Enoncé}}
\subsubsection{Types abstraits}
Dans le cadre de ce projet, nous avions pour consigne de spécifier deux types abstraits : 
\begin{enumerate}
    \item \textbf{Ville} \\
    Ce type abstrait doit permettre de :
    \begin{itemize}
        \item créer une ville à partir de ses deux coordonnées X et Y et de son nom ;
        \item obtenir la coordonnée X d’une escale ;
        \item obtenir la coordonnée Y d’une escale ;
        \item obtenir le nom de la ville ;
        \item calculer la distance géographique entre deux ville ;
        \item enregistrer la spécialité gastronomique de la ville ;
        \item obtenir la spécialité gastronomique de la ville ;
    \end{itemize}
    \item \textbf{Gaule}\\
    Ce type abstrait doit permettre de :
    \begin{itemize}
        \item créer un tour de Gaule sur base de deux villes. Par définition, un tour 
        nouvellement créé ne peut pas constituer un circuit ;
        \item déterminer si un tour de Gaule constitue un circuit ;
        \item déterminer le nombre de villes visitées durant le tour ;
        \item déterminer le nombre totale de spécialités gastronimiques dans le tour ;
        \item déterminer la spécialité gastronomique d’une ville du tour ;
        \item ajouter une ville à un tour ;
        \item supprimer une ville d’un tour ;
    \end{itemize}
\end{enumerate}

\subsubsection{Représentation Concrète}

Ces types abstraits doivent être représentés de 2 manières :
\begin{itemize}
    \item Un tableau
    \item Une liste chainée
\end{itemize}

\section{\textbf{Signature des TAD}}

\subsection{\textbf{Ville}}

\noindent Type :
\begin{itemize}
    \item Ville
\end{itemize}
\noindent Utilise :
\begin{itemize}
    \item $String$
    \item $\mathbb{R}$
\end{itemize}
\noindent Opérations :
\begin{itemize}
    \item creer\_ville : $String \times \mathbb{R} \times \mathbb{R} \rightarrow Ville$
    \item get\_x\_ville : $Ville \rightarrow \mathbb{R}$
    \item get\_y\_ville : $Ville \rightarrow \mathbb{R}$
    \item get\_nom\_ville : $Ville \rightarrow String$
    \item get\_specialite\_ville : $Ville \rightarrow String$
    \item set\_specialite\_ville : $Ville \times String \rightarrow Ville$
    \item distance\_entre\_2\_villes : $Ville \times Ville \rightarrow \mathbb{R}$
\end{itemize}
\noindent Préconditions :
\begin{itemize}
    \item $\emptyset$
\end{itemize}
\noindent Axiomes : $\forall x,y \in \mathbb{R} \wedge \forall V_1, V_2 \in Ville \wedge \forall s,t \in String$
\begin{itemize}
    \item distance\_entre\_2\_villes($V_1,V_2$) = 
    \\$\sqrt{(get\_x\_ville(V_2)-get\_x\_ville(V_1))^2+(get\_y\_ville(V_2)-get\_y\_ville(V_1))^2}$
    \item distance\_entre\_2\_villes(set\_specialite\_ville($V_1, s$), set\_specialite\_ville($V_2, t$)) = \\
    distance\_entre\_2\_villes($V_1,V_2$)
    \item get\_x\_ville(creer\_ville($s, x, y$)) = $x$
    \item get\_y\_ville(creer\_ville($s, x, y$)) = $y$
    \item get\_x\_ville(set\_specialite\_ville($V_1, s$)) = get\_x\_ville($V_1$)
    \item get\_y\_ville(set\_specialite\_ville($V_1, s$)) = get\_y\_ville($V_1$)
    \item get\_nom\_ville(creer\_ville($s, x, y$)) = $s$
    \item get\_nom\_ville(set\_specialite\_ville($V_1, s$)) = get\_nom\_ville($V_1$)
    \item get\_specialite\_ville(set\_specialite\_ville($V_1, s$)) = $s$
    \item get\_specialite\_ville(creer\_ville($s, x, y$)) = NULL

\end{itemize}

\subsection{\textbf{Gaule}}
\noindent Type :
\begin{itemize}
    \item Gaule
\end{itemize}

\noindent Utilise :
\begin{itemize}
    \item $Ville$
    \item $Entiers$
    \item $String$
    \item $Booleen$
\end{itemize}

\noindent Opérations :
\begin{itemize}
    \item cree\_nouveau\_tour : $Ville \times Ville \rightarrow Gaule$
    \item get\_nombre\_villes : $Gaule \rightarrow Entiers$
    \item ajoute\_ville : $Gaule \times Ville \rightarrow Gaule$
    \item supprime\_ville : $Gaule \rightarrow Gaule$
    \item get\_est\_circuit : $Gaule \rightarrow Booleen$
    \item get\_nombre\_specialite : $Gaule \rightarrow Entiers$
    \item get\_specialite : $Gaule \times String \rightarrow String$
    \item ville\_en\_double : $Gaule \times Ville \rightarrow Booleen$
\end{itemize}

\noindent Préconditions :

\noindent Axiomes :$\forall G \in Gaule, \forall V_1, V_2, \in Ville$
\begin{itemize}
    \item get\_nombre\_villes(ajoute\_ville($G, V_1$)) = get\_nombre\_villes($G$)+1
    \item get\_nombre\_villes(supprime\_ville($G$)) = get\_nombre\_villes($G$)-1 
    \\si get\_nombre\_villes($G$)$>0$ 
    \\sinon get\_nombre\_villes(supprime\_ville($G$)) = get\_nombre\_villes($G$)
    \item get\_nombre\_villes(cree\_nouveau\_tour($V_1, V_2$)) = 2
    \item get\_nombre\_specialite(ajoute\_ville($G, V_1$)) = get\_nombre\_specialite($G$)+1
    \\si ville\_en\_double($G, V_1$)=False $\wedge$ get\_specialite($G, V_1$)$\ne$NULL
    \\sinon get\_nombre\_specialite(ajoute\_ville($G, V_1$)) = get\_nombre\_specialite($G$)
    \item get\_nombre\_specialite(cree\_nouveau\_tour($V_1, V_2$)) = 2
    \\si get\_specialite($G, V_1$)$\ne$NULL $\wedge$ get\_specialite($G, V_2$)$\ne$NULL
    \item get\_nombre\_specialite(cree\_nouveau\_tour($V_1, V_2$)) = 1
    \\si get\_specialite($G, V_1$)$\ne$NULL $\wedge$ get\_specialite($G, V_2$)=NULL
    \\$\vee$get\_specialite($G, V_2$)$\ne$NULL $\wedge$ get\_specialite($G, V_1$)=NULL
    \item get\_nombre\_specialite(cree\_nouveau\_tour($V_1, V_2$)) = 0
    \\si get\_specialite($G, V_1$)=NULL $\wedge$ get\_specialite($G, V_2$)=NULL
    \item get\_est\_circuit(cree\_nouveau\_tour($V_1, V_2$)) = False
    \item get\_specialite(cree\_nouveau\_tour($V_1, V_2$), $V_1$) =  get\_specialite\_ville($V_1$)
    \item get\_specialite(cree\_nouveau\_tour($V_1, V_2$), $V_2$) =  get\_specialite\_ville($V_2$)
    \item get\_specialite(ajoute\_ville($G, V_1$), $V_1$) = get\_specialite\_ville($V_1$)
\end{itemize}

\section{\textbf{Spécifications des TAD}}

\subsection{\textbf{Ville}}
\subsubsection{Structure}
\begin{lstlisting}[caption = {Structure "Ville"}]
    struct Ville_t{
    char *nom;
    float x;
    float y;
    char *specialite;
    };
\end{lstlisting}

\subsubsection{Spécification des fonctions et procédures de ville.h}
\begin{lstlisting}[caption = {Spécification des fonctions et procédures du header "ville.h"}]
    /*
    *@pre : (>\color{MidnightBlue}$nom \ne NULL \wedge x=x_0\wedge y=y_0 \wedge nom=nom_0$<)
    *@post : (>\color{MidnightBlue}$ville_{init}\wedge  x=x_0\wedge y=y_0 \wedge nom=nom_0\wedge get\_x\_ville(ville)=ville->x\wedge$<)
    *(>\color{MidnightBlue}$get\_y\_ville(ville)=ville->y \wedge get\_nom\_ville(ville)=ville->nom$<)
    */
    Ville *creer_ville(char *nom, float x, float y);

    /*
    *@pre : (>\color{MidnightBlue}$\emptyset$<)
    *@post : (>\color{MidnightBlue}$ville=NULL$<)
    */
    void detruit_ville(Ville *ville);
    
    /*
    *@pre : (>\color{MidnightBlue}$ville\ne NULL$<)
    *@post : (>\color{MidnightBlue}$ville=ville_0 \wedge get\_x\_ville(ville)=ville->x$<)
    */
    float get_x_ville(Ville *ville);
    
    /*
    *@pre : (>\color{MidnightBlue}$ville\ne NULL$<)
    *@post : (>\color{MidnightBlue}$ville=ville_0 \wedge get\_y\_ville(ville)=ville->y$<)
    */
    float get_y_ville(Ville *ville);
    
    /*
    *@pre : (>\color{MidnightBlue}$ville\ne NULL$<)
    *@post : (>\color{MidnightBlue}$ville=ville_0 \wedge get\_nom\_ville(ville)=ville->nom$<)
    */
    char *get_nom_ville(Ville *ville);
    
    /*
    *@pre : (>\color{MidnightBlue}$ville\ne NULL$<)
    *@post : (>\color{MidnightBlue}$ville=ville_0 \wedge get\_specialite\_ville(ville)=ville->specialite$<)
    */
    char *get_specialite_ville(Ville *ville);
    
    /*
    *@pre : (>\color{MidnightBlue}$ville\ne NULL \wedge specialite\ne NULL\wedge specialite = specialite_0$<)
    *@post : (>\color{MidnightBlue}$ville=ville_0\wedge specialite = specialite_0 \wedge get\_specialite\_ville(ville)=specialite$<)
    */
    void set_specialite_ville(Ville *ville, char *specialite);
    
    /*
    *@pre : (>\color{MidnightBlue}$ville1=ville1_0\ne NULL \wedge ville2=ville2_0\ne NULL $<)
    *@post : (>\color{MidnightBlue}$ville1=ville1_0 \wedge ville2=ville2_0\wedge $<)
    *(>\color{MidnightBlue}$distance\_entre\_2\_villes(ville1,ville2) = $<)
    *(>\color{MidnightBlue}$\sqrt{(get\_x\_ville(ville2)-get\_x\_ville(ville1))^2+(get\_y\_ville(ville2)-get\_y\_ville(ville1))^2}$<)
    */
    float distance_entre_2_villes(Ville *ville1, Ville *ville2);
    
    /*
    *@pre : (>\color{MidnightBlue}$\emptyset$<)
    *@post : (>\color{MidnightBlue}retourne la taille mémoire de la struct Ville<)
    */
    int size_ville(void);
\end{lstlisting}

\subsection{\textbf{Gaule}}
\subsubsection{Structure en Tableau}
\begin{lstlisting}[caption = {Structure "Gaule" dans l'implémentation en tableau}]
    struct Gaule_t{
    Ville **tableau_ville;
    int nombre_villes;
    int est_circuit;
    int nombre_specialites;
    };
\end{lstlisting}



\subsubsection{Structure en Liste chainée}
    Pour la liste chainée, une deuxième structure vient s'ajouter. La première, comme pour les tableaux, 
    garde les informations sur la liste et la deuxième sont les structures qui correspondront chacune à une 
    des villes avec un pointeur sur l'élément suivant et précédent de la liste.
\begin{lstlisting}[caption = {Structure "Gaule" dans l'implémentation en tableau}]
    struct Gaule_t{
    Cellule_Gaule *premiere_cellule;
    Cellule_Gaule *derniere_cellule;
    int nombre_villes;
    int est_circuit;
    int nombre_specialites;
    };

    struct Cellule_Gaule_t{
    Cellule_Gaule *cellule_suivante;
    Cellule_Gaule *cellule_precedente;
    Ville *ville;
    };
\end{lstlisting}

\subsubsection{Spécification des fonctions et procédures gaule.h}















\end{document}